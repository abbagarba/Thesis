% Copyright (c) 2014,2016 Casper Ti. Vector
% Public domain.

\specialchap{总结与展望}



PKI技术作为能够实现身份鉴别、机密性、完整性、非否认等核心安全服务的基础设施, 在信息系统安全中发挥着重要作用。本文将区块链技术应用于现有的PKI系统中,提出了一种针对于Web PKI中域名实体的身份认证方案,解决了原有PKI体系下信任构建集中化的问题。

通过对现有PKI系统的分析,可以得知所有的信任都是以中心化的CA作为支点,这样做大大减弱了现有系统设计的复杂度,使得
PKI被广泛的部署到日常生活和商业应用当中。中心化的CA作为PKI系统的核心,拥有以上优势的同时也存在相应的弊端和潜在的危险,导致了本系统的权利分配不均并且容易受到CA不端操作的影响。本文第三章对PKI中存在的其它问题也进行描述,并对现有的相关解决方案进行了归纳和总结。


本文给出方案的主要思想是将PKI系统中用户信任的CA列表通过区块链来记录,拥有公开且不可篡改的特性。与具有类似思路的已有方案是基于DNS的授权实体命名(DANE)方案,该方案是将域名的信任CA列表记录到DNS服务器上,但该方案中的身份认证工作依赖于DNS的身份管理服务,并且其安全性严重依赖于DNS服务的安全。与之相比,本文给出基于区块链的身份认证协议,其不需要依赖可信第三方来完成,而是通过区块链网络中的节点进行身份的验证和确认;同时,所有的操作和内容都记录在区块链上,由于其不可篡改的特性更加安全有效。

在完成方案设计之后,对方案进行了分析,论证了方案的可行性和安全性;其后对方案的实现方式进行了讨论,可以基于智能合约完成或是通过开发新的链来完成;本选选择较为简单的基于智能合约的方式进行了系统设计和实现,对实现的各个模块进行了功能性测试。


本给出的方案中涉及到很多参数的选择,如挑选验证节点的难度调整时间、验证的时间、验证的次数等,对于不同的实现环境和安全强度将会有不同设置,下一步的工作应该对这些参数的设置给出相关的理论分析和相关建议;同时,本文中给出了基于以太坊上智能合约的实现,如果部署到真实的环境当中,由于以太坊自身效率的问题,其认证效率会大大折扣,并且需要付出较高的代价;最后,本文在实现过程中给出域名客户端和验证节点客户端没有图形化的界面,需要进一步完善客户端的完整性,才能简单便捷的被投入使用。

将以上提到的工作进一步完成之后,本文给出的基于区块链的方案实际上已经实现了对域名身份的认证,完成认证的公私钥对完全可以用在安全通信的建立,取代现有的PKI系统,实现一个基于区块链的公钥基础设施;该基础设施将不需要有可信第三方,完全依靠网络中的用户来完成身份的认证,并且具有公开、透明、防篡改的特性,但是达到这种状态还需要很长一段时间的发展,以及可能需要更好的去中心化的身份认证方案被提出。



%代价or速度?




% vim:ts=4:sw=4
