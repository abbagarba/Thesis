% Copyright (c) 2014,2016 Casper Ti. Vector
% Public domain.

\begin{cabstract}




%随着网络技术的快速发展,计算机应用逐渐渗透到日常生活中的各个方面,利用网络完成各种信息的交换和处理已经成为日常生活的一部分。在电子邮件、电子商务、网上银行及资源发布等日常网络应用中,需要在开放网络环境下与不明身份实体之间进行通信,如何保证在数据传输过程中的机密性、完整性、身份鉴别和不可抵赖性一直备受大家关注。公钥密码体系的出现,特别是公钥基础设施(PKI)技术的发展,是解决以上问题的重要方式。经过近二十年的发展,PKI相关的技术和标准逐渐完善,各类CA认证中心相继被建立,在以上提及的应用中得到普及。

%第一段,为什么要做这个事情
公钥基础设施(PKI)作为保证数据机密性、完整性、不可抵赖性以及身份鉴别的重要方式,已经被广泛应用在电子邮件、电子商务、网上银行及资源发布等日常网络应用。经过近二十年的发展,PKI相关技术和标准逐渐完善,但由于复杂的应用环境和中心化的系统架构,任然存在着各种问题。这些问题主要表现在系统的信任构建和签发证书的可信度两方面,其中签发证书的可信度问题可以通过一些列外在的改进措施,规范证书操作的流程来解决;系统的信任构建问题需要对现有PKI系统组件和架构进行调整,均衡各个实体之间的信任和权利分配,近年来一直是PKI相关研究所关注的问题。

% 本文通过对现有PKI系统的介绍和分析,根据证书的生命周期透视该系统中可能存在的问题,并对现有的解决方案进行阐述和归纳。针对web PKI系统中信任构建问题,本文提出了一种可以提升域名对证书签发控制权的方案;该方案给出了一套基于区块链的身份认证协议,让域名可以在去中心化环境中自主选择信任的CA,从而减弱现有系统中授权机构对证书签发的绝对控制权,构建更加信任更加分散的PKI系统。本文对给出方案的安全性进行了理论分析,并基于以太坊上的智能合约进行了实现。

%做了什么调研,分析了什么,解决了什么
%做了什么调研
本文对现有的PKI系统架构进行介绍和分析,依照证书的生命周期透视该系统中可能存在的问题,并对现有的解决方案进行分析和归类。其后对具有去中心化特性的区块链技术进行介绍,包括共识机制和智能合约技术,并调研了与区块链相关的PKI解决方案。
%分析了什么
发现现有PKI系统中的信任构建问题,其存在的主要原因是该系统赋予中心化的授权机构(CA)权力过高。为了削弱CA在PKI系统中对证书的控制力,需要均衡各个实体之间的权利分配,提升证书申请者对证书签发的控制权。
%解决了什么
本文基于区块链技术提出了一种域名身份认证的方案,域名拥有者可以在分布式、无需可信第三方的网络环境下完成对信任CA的控制;该方案作为现有PKI系统的辅助工具,可以帮助PKI用户提升对证书签发的控制权,削弱授权中心的绝对权力,增强PKI系统的安全性。

%本文的工作
与现有的类似方案DANE相比,本方案利用区块链技术完成身份认证,无需可信第三方,具有更好的安全性;另外,本方案中的所有操作在区块链上进行,提供了公开透明的审计特性。本文就给出方案的安全性进行了分析,对可能存在的攻击方式进行了讨论,并给出其免受这些攻击的说明。最后,对本文提出的方案进行了系统设计,详细描述了相关模块的功能和接口;依据设计给出相应的系统实现,对实现的各个模块进行功能性测试,系统能够良好稳定地运行;并对未来能够开展的工作和适用的应用场景进行了分析。

% 1.用一句话或一个短语介绍要解决的问题;

% 2.解释现有方法或已有研究的不足之处;

% 3. 重点写出 :  清楚地描述你的主要贡献 ,当然也可以把这个放在开头;
 
% 4. 用两三句话来说明细节,以及主要的定量结果等等,还可以有一句展望

% %目的
% 背景知识介绍。
% 公钥密码学的广泛应用使得公钥真实性的可验证性成为必要条件。而公钥基础设施(PKI)作为当前最为全面的信息安全解决方案,是利用证书机制来传递公钥间的相互信任的。

% 随着电子邮件、电子商务、网上银行及资源发布等Internet和Intranet应用的发展,经常需要在开放网络中的不明身份实体之间进行通信,为了保障网络上的各种应用的机密性、完整性、身份鉴别和不可抵赖性。

% 经过多年的研究,初步形成一套完整的Internet安全解决方案,即目前被广泛采用的PKI技术(Public Key Infrastructure—公钥基础设施)。 PKI技术采用证书管理公钥,通过第三方的可信任机构——认证中心CA(Certificate Authority)把用户的公钥和用户的其他标识信息(如名称、E-mail、身份证号等)捆绑在一起。通过CA机构,较好的解决了密钥分发和管理问题,并通过数字证书,对传输的数据进行加密和鉴别,保证了信息传输的机密性、真实性、完整性和不可否认性。 
% 公钥基础设施PKI以非对称加密技术为基础,为网络信息安全提供有力保障。PKI以数字证书为密钥管理工具,终端实体之间进行通信之前必须要验证数字证书的可信性。验证数字证书是否可信的过程中,首先需要构造一条连接通信双方的证书路径,通过验证该证书路径所有证书的可信性,来最终确定通信目的方证书的可信性。

% 随着全球计算机互联网络用户覆盖范围和信息传输量的迅速发展,各类网络应用也日益增多。人们的社会活动和经济活动越来越依赖于计算机网络,因而网络的安全性已成为信息化建设的一个核心问题。以PKI(Public Key Infrastructure,公开密钥基础设施)为基础的安全认证平台是使互联网应用向商贸领域发展成为可能的必要条件,它保证了端到端的安全,主要用于支持身份认证、数据完整性保护、数据保密性、不可否认性和不可伪造性等安全服务。

% 电子商务、电子政务等基于Internet的网络增值应用日新月异,这些应用对信息安全的需求也随之提升,诸如公平性、可追踪性等安全特性就是除了传统的保密性、完整性、非否认性、身份认证等基本安全要求之外的新需求。基于公钥密码技术构建的公钥基础设施(PKI)是目前公认的解决大型开放网络环境下信息安全问题最可行、最有效的办法。

% 随着计算机应用对社会生活各方面的渗透,利用计算机进行各种信息处理已越来越成为一种趋势。由于这些信息多涉及到国家的军事、政治、经济秘密以及一些个人隐私,所以计算机安全越来越受重视。计算机安全涉及加密、认证、访问控制、入侵检测等多个方面。作为信息安全基础的公钥基础设施(PKI)已经成为当今计算机安全领域研究的热点问题。

%方法
%m本文通过查找参考文献等途径,先对什么进行了分析,并在此基础上,通过什么方法对什么进行了研究,最后得出了什么结论


%结果和结论

\end{cabstract}




\begin{eabstract}
Public Key Infrastructure (PKI) has been widely used in daily network applications, such as e-mail, e-commerce, online banking, and resource publishing, which is an important way to ensure data confidentiality, integrity, non-repudiation, and identity authentication. After nearly two decades of development, PKI-related technologies and standards have gradually improved. But due to the complex application environment and centralized system architecture, there are still various problems. These problems are mainly manifested in the trust building of the system and the credibility of the issuance certificate. The credibility of the issuance certificate can be solved by some external improvement measures and the process of standardizing the certificate operations. The trust construction problem of this system needs to adjust existing system components and architecture to balance the trust and rights allocation between entities, and it has been a concern of PKI-related research institutes in recent years.

This paper introduces and analyzes the existing architecture of PKI system, analyzes the possible problems in the system according to the life cycle of the certificate, and then analyzes and categorizes the existing solutions. Then it introduces the blockchain technology with decentralized characteristics, including consensus mechanism and smart contract technology, and investigates the PKI solution related to blockchain. The main reason for the existence of the trust-building problem in the existing PKI system is that the system gives the Certificate Authority (CA) power to be too high. In order to weaken the CA's control over certificates, it is necessary to balance the distribution of rights among entities and improve the control of certificate issuers for certificate owner. This paper proposes a domain name authentication scheme based on blockchain technology. The domain owner can complete the control of the trusted CA in a distributed, trusted third party network environment; this solution is used as an auxiliary tool for the existing PKI system, which can help PKI users to improve the control over the issuance of certificates, weaken the absolute authority of the authorized center, and enhance the security of the PKI system.

Compared with the existing solutions like DANE, this scheme uses blockchain technology to identity authentication without a trusted third party. Its security does not depend on other application entities; in addition, all operations are completed through the blockchain, which provides open and transparent auditing features. This paper analyzes the possible attacks to our scheme, and gives explanation for away from these attacks. Furtherly, the solution proposed in this paper is implemented. The functions and interfaces of related modules are described in detail. According to the design, the functional testing of each module is performed, and the system can operate in a good and stable manner. At Last, future work and applicable scenarios are discussed.

\end{eabstract}

% vim:ts=4:sw=4
